\pdfoutput=1
\documentclass[red,handout,professionalfont]{beamer}
% \documentclass[red,professionalfont]{beamer}
\usepackage{multimedia}
\usepackage{chessboard}
\usepackage{url}
\usepackage{amssymb}  % the check symbol 
% Python listing setup

\usepackage{color}
\usepackage[procnames]{listings}
\usepackage{textcomp}
\usepackage{setspace}
\usepackage{palatino}
\renewcommand{\lstlistlistingname}{Code Listings}
\renewcommand{\lstlistingname}{Code Listing}
\definecolor{gray}{gray}{0.5}
\definecolor{green}{rgb}{0,0.5,0}
\definecolor{lightgreen}{rgb}{0,0.7,0}
\definecolor{purple}{rgb}{0.5,0,0.5}
\definecolor{darkred}{rgb}{0.5,0,0}
\lstnewenvironment{python}[1][]{
\lstset{
language=python,
breaklines=true,
basicstyle=\ttfamily\small\setstretch{1},
stringstyle=\color{green},
showstringspaces=false,
alsoletter={1234567890},
otherkeywords={\ , \}, \{},
keywordstyle=\color{blue},
emph={access,and,as,break,class,continue,def,del,elif,else,%
except,exec,finally,for,from,global,if,import,in,is,%
lambda,not,or,pass,print,raise,return,try,while,assert},
emphstyle=\color{orange}\bfseries,
emph={[2]self},
emphstyle=[2]\color{gray},
emph={[4]ArithmeticError,AssertionError,AttributeError,BaseException,%
DeprecationWarning,EOFError,Ellipsis,EnvironmentError,Exception,%
False,FloatingPointError,FutureWarning,GeneratorExit,IOError,%
ImportError,ImportWarning,IndentationError,IndexError,KeyError,%
KeyboardInterrupt,LookupError,MemoryError,NameError,None,%
NotImplemented,NotImplementedError,OSError,OverflowError,%
PendingDeprecationWarning,ReferenceError,RuntimeError,RuntimeWarning,%
StandardError,StopIteration,SyntaxError,SyntaxWarning,SystemError,%
SystemExit,TabError,True,TypeError,UnboundLocalError,UnicodeDecodeError,%
UnicodeEncodeError,UnicodeError,UnicodeTranslateError,UnicodeWarning,%
UserWarning,ValueError,Warning,ZeroDivisionError,abs,all,any,apply,%
basestring,bool,buffer,callable,chr,classmethod,cmp,coerce,compile,%
complex,copyright,credits,delattr,dict,dir,divmod,enumerate,eval,%
execfile,exit,file,filter,float,frozenset,getattr,globals,hasattr,%
hash,help,hex,id,input,int,intern,isinstance,issubclass,iter,len,%
license,list,locals,long,map,max,min,object,oct,open,ord,pow,property,%
quit,range,raw_input,reduce,reload,repr,reversed,round,set,setattr,%
slice,sorted,staticmethod,str,sum,super,tuple,type,unichr,unicode,%
vars,xrange,zip},
emphstyle=[4]\color{purple}\bfseries,
upquote=true,
morecomment=[s][\color{lightgreen}]{"""}{"""},
commentstyle=\color{red}\slshape,
literate={>>>}{\textbf{\textcolor{darkred}{>{>}>}}}3%
         {...}{{\textcolor{gray}{...}}}3,
procnamekeys={def,class},
procnamestyle=\color{blue}\textbf,
framexleftmargin=1mm, framextopmargin=1mm,% frame=shadowbox,
%rulesepcolor=\color{blue},
#1
}}{}


\usepackage{tikz}
\usepackage{tikz-qtree}
\usetikzlibrary{decorations.pathreplacing,positioning}
%\bibliography{mujstyl}
\theoremstyle{definition}
\newtheorem{definice}{Definition}[section]
\newtheorem{thm}{Theorem}[section]
\newtheorem{idea}{Idea}[section]
\newtheorem{cor}[thm]{Corrollary}
\newtheorem{lem}[thm]{Lemma}
\newtheorem{obs}[thm]{Observation}
\newtheorem{rem}[thm]{Remark}
\newtheorem{ex}[thm]{Example}
\newtheorem{quizz}[thm]{Question}
\newcommand{\pomega}{\mbox{$\mathcal{P}(\omega)$}}
\newcommand{\cont}{\mbox{$\mathfrak c$}}
\newcommand{\ba}{\mbox{${\mathbb B}$}}
\newcommand{\0}{\mbox{${\bf 0}$}}
\newcommand{\F}{\mbox{${\mathcal F}$}}
\newcommand{\rest}{\mbox{$\upharpoonright$}}
\newcommand{\cl}[1]{\mbox{$\overline{#1}$}}
\newcommand{\yes}{\textcolor{green}{$\checkmark$}}
\newcommand{\no}{\textcolor{red}{$\times$}}
\renewcommand{\emph}[1]{{\bf #1}}
\mode<presentation>
{
\useinnertheme{rounded}

\usecolortheme{whale}
\usecolortheme{orchid}

\setbeamerfont{block title}{size={}}

%   \useoutertheme{default}
%   \usetheme{Copenhagen}
%   \useoutertheme{default}
  \setbeamercovered{invisible}
}
\setbeamertemplate{navigation symbols}{} 
\usepackage[utf8]{inputenc}
\usepackage[czech,english]{babel}
\usepackage{lmodern}
%\usepackage{times}
\usepackage[T1]{fontenc}


\tikzset{onslide/.code args={<#1>#2}{%
  \only<#1>{\pgfkeysalso{#2}} % \pgfkeysalso doesn't change the path
}}

\tikzstyle{hilight}=[red,ultra thick]
\tikzstyle{active}=[yellow,ultra thick]
\tikzstyle{memory}=[blue]



\title[]{Zpracování obrazu\\ (10. přednáška)}

% Dnes se podíváme na jednoduché "goal-based" agenty
% a ukážeme si obecné postupy 


% \author[]{Jonathan L. Verner}
% \institute[Charles University, Prague] % (optional,but mostly needed)
% {
%   Department of Logic\\
  %   Faculty of Philosophy\\
%   Charles University in Prague
% }
\date[]{}
% \subject{}
%\pgfdeclareimage[height=1cm]{university-logo}{UK-logo}
%\logo{\pgfuseimage{university-logo}}

\begin{document}

\AtBeginSection[]
{
  \begin{frame}<beamer>
    \begin{block}{}
%    \begin{centering}
    \hfill\insertsection\hfill\
 %   \end{centering}
    \end{block}
    %\tableofcontents[currentsection]
  \end{frame}
}

\AtBeginSubsection[]
{
  \begin{frame}<beamer>
    \begin{block}{}
  %  \begin{centering}
    \hfill\insertsubsection\hfill\
   % \begin{centering}
    \end{block}
    %\tableofcontents[currentsection]
  \end{frame}
}

%#################################################################

\begin{frame}{} \titlepage
%{\ \hfill \includegraphics[width=1cm]{UK-logo}\hfill\ }
\end{frame}

\begin{frame}\frametitle{Od reality k obrazu $\ldots$}
\begin{center}
\begin{minipage}{4cm}
\includegraphics[width=4cm]{vision/mesh_cup.png}
\begin{displaymath}
 W
\end{displaymath}
\end{minipage}\pause
\begin{minipage}{1cm}
\begin{displaymath}
 \Rightarrow
\end{displaymath}
\end{minipage}\pause
\begin{minipage}{4cm}
\includegraphics[width=4cm]{vision/real_cup.png}
\begin{displaymath}
 g(W)
\end{displaymath}
\end{minipage}
\end{center}
\end{frame}

\begin{frame}{$\ldots$ a zase zpět}
\begin{center}
 A co naopak?
\end{center}\pause
\begin{displaymath}
 g(W)\Rightarrow W
\end{displaymath}\pause
\begin{center}
Nejednoznačnost
\end{center}\pause
\begin{center}
\alt<4>{\includegraphics[width=8cm]{vision/ambiguity1.pdf}}{\includegraphics[width=8cm]{vision/ambiguity2.pdf}}
\end{center}\pause
\end{frame}


\begin{frame}\frametitle{Zjednodušený model --- camera obscura}
\begin{center}
 \includegraphics[width=5cm]{vision/camera_obscura.pdf}
\end{center}\pause
\begin{itemize}
\item[$P$] --- bod na scéně, souřadnice $(X,Y,Z)$\pause
\item[$P'$] --- obraz bodu $P$ v rovině obrázku, souřadnice $(x,y,z)$\pause
\end{itemize}
Pomocí podobnosti trojúhelníků odvodíme\pause
\begin{displaymath}
x=\frac{-fX}{Z},\ y=\frac{-fY}{Z} 
\end{displaymath}
\end{frame}

\begin{frame}\frametitle{Intenzita, barvy, $\ldots$}
\begin{block}{}
\begin{center} 
Intenzita závisí na úhlu dopadu světla\pause\\
(Lambertův zákon)
\end{center}\pause
\end{block}
\begin{displaymath}
 I = \varrho_0I_0\cos\vartheta
\end{displaymath}\pause
\begin{itemize}
\item[$I_0$] --- intenzita dopadajícího světla\pause
\item[$\varrho_0$] --- albedo\pause
\item[$\vartheta$] --- úhel dopadu
\end{itemize}\pause
\begin{block}{}
\begin{center}
Princip trojbarevnosti
\end{center}
\end{block}\pause
\begin{center}
Každé vlnové spektrum lze složit ze tří složek tak, že člověk nepozná rozdíl.
\end{center}\pause

\begin{itemize}
 \item složky: červená, zelená, modrá\pause
 \item ústřice mají 12 základních barev (!)
\end{itemize}
\end{frame}

\begin{frame}\frametitle{Fáze zpracování obrazu}
\begin{center}
\alert{Nízkourovňová}\pause
\end{center}
\begin{itemize}
 \item detekce hran\pause
 \item detekce oblastí\pause
 \item jednoduchá detekce objektů\pause
 \item 2D $\Rightarrow$ 3D\pause
\end{itemize}
\begin{center}
\alert{Vysokoúrovňová}\pause
\end{center}
\begin{itemize}
 \item rozpoznávání objektů\pause
 \item rozpoznávání textu\pause
 \item $\ldots$
\end{itemize}
\end{frame}

\begin{frame}\frametitle{Detekce hran}
\pause
\begin{center}
\begin{minipage}{4cm}
\includegraphics[width=4cm]{vision/stapler.pdf}
\end{minipage}\pause
\begin{minipage}{1cm}
\begin{displaymath}
 \Rightarrow
\end{displaymath}
\end{minipage}\pause
\begin{minipage}{4cm}
\includegraphics[width=4cm]{vision/stapler-edges.pdf}
\end{minipage}
\end{center}\pause
\begin{block}{}
\begin{center}
 Idea\pause\\
 Hrana je místo, kde dochází k ostré změně intenzity.
 \end{center}
\end{block}\pause
\begin{itemize}
 \item derivace intenzity, gradient intenzity \pause
 \item odstranění šumu (gausovský filtr)\pause
 \item konvoluce\pause
\end{itemize}
\end{frame}

\begin{frame}\frametitle{Detekce hran podruhé}
\begin{center}
 U strukturovaných povrchů (vlasy, oblečení, ...) převažují lokální změny intenzit
 nad globálními
\end{center}\pause
\begin{itemize}
 \item otisky textur --- histogram intenzit
 \item relativně nezávislé na změně osvětlení
\end{itemize}\pause
\end{frame}

\begin{frame}\frametitle{Optický tok}
\begin{center}
 \includegraphics[width=9cm]{vision/tennis.pdf}
\end{center}
\end{frame}

\begin{frame}\frametitle{Detekce oblastí}
\begin{center}
 \includegraphics[width=9cm]{vision/segmentation.pdf}
\end{center}
\end{frame}

\begin{frame}\frametitle{Detekce obličejů}
\end{frame}

\begin{frame}\frametitle{2D$\Rightarrow$3D}
\begin{center}
\includegraphics[width=9cm]{vision/binocular.pdf}
\end{center}
\end{frame}


\begin{frame}\frametitle{2D$\Rightarrow$3D}
\begin{center}
\includegraphics[width=9cm]{vision/scenery.pdf}
\end{center}
\end{frame}

\begin{frame}\frametitle{2D$\Rightarrow$3D}
\begin{center}
\includegraphics[width=9cm]{vision/horizont.pdf}
\end{center}
\end{frame}


\end{document}
\pdfoutput=1
% \documentclass[red,handout,professionalfont]{beamer}
\documentclass[red,professionalfont]{beamer}
\usepackage{multimedia}
\usepackage{url}
\usepackage{amssymb}  % the check symbol 
% Python listing setup

\usepackage{color}
\usepackage[procnames]{listings}
\usepackage{textcomp}
\usepackage{setspace}
\usepackage{palatino}
\renewcommand{\lstlistlistingname}{Code Listings}
\renewcommand{\lstlistingname}{Code Listing}
\definecolor{gray}{gray}{0.5}
\definecolor{green}{rgb}{0,0.5,0}
\definecolor{lightgreen}{rgb}{0,0.7,0}
\definecolor{purple}{rgb}{0.5,0,0.5}
\definecolor{darkred}{rgb}{0.5,0,0}
\lstnewenvironment{python}[1][]{
\lstset{
language=python,
breaklines=true,
basicstyle=\ttfamily\small\setstretch{1},
stringstyle=\color{green},
showstringspaces=false,
alsoletter={1234567890},
otherkeywords={\ , \}, \{},
keywordstyle=\color{blue},
emph={access,and,as,break,class,continue,def,del,elif,else,%
except,exec,finally,for,from,global,if,import,in,is,%
lambda,not,or,pass,print,raise,return,try,while,assert},
emphstyle=\color{orange}\bfseries,
emph={[2]self},
emphstyle=[2]\color{gray},
emph={[4]ArithmeticError,AssertionError,AttributeError,BaseException,%
DeprecationWarning,EOFError,Ellipsis,EnvironmentError,Exception,%
False,FloatingPointError,FutureWarning,GeneratorExit,IOError,%
ImportError,ImportWarning,IndentationError,IndexError,KeyError,%
KeyboardInterrupt,LookupError,MemoryError,NameError,None,%
NotImplemented,NotImplementedError,OSError,OverflowError,%
PendingDeprecationWarning,ReferenceError,RuntimeError,RuntimeWarning,%
StandardError,StopIteration,SyntaxError,SyntaxWarning,SystemError,%
SystemExit,TabError,True,TypeError,UnboundLocalError,UnicodeDecodeError,%
UnicodeEncodeError,UnicodeError,UnicodeTranslateError,UnicodeWarning,%
UserWarning,ValueError,Warning,ZeroDivisionError,abs,all,any,apply,%
basestring,bool,buffer,callable,chr,classmethod,cmp,coerce,compile,%
complex,copyright,credits,delattr,dict,dir,divmod,enumerate,eval,%
execfile,exit,file,filter,float,frozenset,getattr,globals,hasattr,%
hash,help,hex,id,input,int,intern,isinstance,issubclass,iter,len,%
license,list,locals,long,map,max,min,object,oct,open,ord,pow,property,%
quit,range,raw_input,reduce,reload,repr,reversed,round,set,setattr,%
slice,sorted,staticmethod,str,sum,super,tuple,type,unichr,unicode,%
vars,xrange,zip},
emphstyle=[4]\color{purple}\bfseries,
upquote=true,
morecomment=[s][\color{lightgreen}]{"""}{"""},
commentstyle=\color{red}\slshape,
literate={>>>}{\textbf{\textcolor{darkred}{>{>}>}}}3%
         {...}{{\textcolor{gray}{...}}}3,
procnamekeys={def,class},
procnamestyle=\color{blue}\textbf,
framexleftmargin=1mm, framextopmargin=1mm,% frame=shadowbox,
%rulesepcolor=\color{blue},
#1
}}{}


\usepackage{tikz}
\usetikzlibrary{decorations.pathreplacing}
%\bibliography{mujstyl}
\theoremstyle{definition}
\newtheorem{definice}{Definition}[section]
\newtheorem{thm}{Theorem}[section]
\newtheorem{idea}{Idea}[section]
\newtheorem{cor}[thm]{Corrollary}
\newtheorem{lem}[thm]{Lemma}
\newtheorem{obs}[thm]{Observation}
\newtheorem{rem}[thm]{Remark}
\newtheorem{ex}[thm]{Example}
\newtheorem{quizz}[thm]{Question}
\newcommand{\pomega}{\mbox{$\mathcal{P}(\omega)$}}
\newcommand{\cont}{\mbox{$\mathfrak c$}}
\newcommand{\ba}{\mbox{${\mathbb B}$}}
\newcommand{\0}{\mbox{${\bf 0}$}}
\newcommand{\F}{\mbox{${\mathcal F}$}}
\newcommand{\rest}{\mbox{$\upharpoonright$}}
\newcommand{\cl}[1]{\mbox{$\overline{#1}$}}
\newcommand{\yes}{\textcolor{green}{$\checkmark$}}
\newcommand{\no}{\textcolor{red}{$\times$}}
\renewcommand{\emph}[1]{{\bf #1}}
\mode<presentation>
{
\useinnertheme{rounded}

\usecolortheme{whale}
\usecolortheme{orchid}

\setbeamerfont{block title}{size={}}

%   \useoutertheme{default}
%   \usetheme{Copenhagen}
%   \useoutertheme{default}
  \setbeamercovered{invisible}
}
\setbeamertemplate{navigation symbols}{} 
\usepackage[utf8]{inputenc}
\usepackage[czech,english]{babel}
\usepackage{lmodern}
%\usepackage{times}
\usepackage[T1]{fontenc}


\title[]{Agenti \\ (2. přednáška)}
% \author[]{Jonathan L. Verner}
% \institute[Charles University, Prague] % (optional,but mostly needed)
% {
%   Department of Logic\\
%   Faculty of Philosophy\\
%   Charles University in Prague
% }
\date[]{}
% \subject{}
%\pgfdeclareimage[height=1cm]{university-logo}{UK-logo}
%\logo{\pgfuseimage{university-logo}}

\begin{document}

\AtBeginSection[]
{
  \begin{frame}<beamer>
    \begin{block}{}
%    \begin{centering}
    \hfill\insertsection\hfill\
 %   \end{centering}
    \end{block}
    %\tableofcontents[currentsection]
  \end{frame}
}

\AtBeginSubsection[]
{
  \begin{frame}<beamer>
    \begin{block}{}
  %  \begin{centering}
    \hfill\insertsubsection\hfill\
   % \begin{centering}
    \end{block}
    %\tableofcontents[currentsection]
  \end{frame}
}

%#################################################################

\begin{frame}{} \titlepage
%{\ \hfill \includegraphics[width=1cm]{UK-logo}\hfill\ }
\end{frame}

\begin{frame}\frametitle{Návrh Racionálního Agenta}
\begin{block}{}
\begin{displaymath}
 f: {\mathcal P}^* \to {\mathcal A}
\end{displaymath}\pause
\begin{center}
 Je třeba specifikovat
\end{center}\pause
\end{block}
\begin{itemize}
 \item možné vjemy (množina $\mathcal P$)\pause
 \item možné akce (množina $\mathcal A$)\pause
 \item \alert{prostředí}\pause
 \item měřítko úspěšnosti
\end{itemize}
\end{frame}

\begin{frame}[fragile]\frametitle{Příklady}
\begin{center}
{%\tiny
\hskip-1cm
\begin{tabular}{p{1.9cm}|p{2cm}|p{2cm}|p{2cm}|p{1cm}}
 {\bf Agent } & {\bf Vjemy }& {\bf Akce }& {\bf Prostředí } &{\bf Úspěšnost} \\[0.1cm]\pause
 hráč šachů\pause & tahy protihráče\pause, stav šachovnice\pause & vlastní tahy\pause & Šachovnice\pause & ELO \\[0.5cm]\pause
 hráč mariáše\pause & tahy protihráče\pause, karty v ruce\pause & vlastní tahy\pause & hospoda\pause & ? \\[0.5cm]\pause
 autonomní vozidlo\pause & video obraz vozovky\pause, rychlost\pause, GPS\pause, $\ldots$\pause & přidání plynu\pause,  otočení volantem\pause , $\ldots$\pause & vozovka\pause & bezpečnost\pause, spolehlivost\pause, rychlost\cr
 automatický vysavač\pause & senzor špíny\pause & vysávání, pohyb\pause & byt, kancelář& čistota\pause, hlučnost\pause,vysátá špína\cr
 \end{tabular}
}
\end{center}

\end{frame}

\begin{frame}\frametitle{Klasifikace Prostředí}
 \begin{itemize}
  \item plná vs. částečná pozorovatelnost\pause
  \begin{itemize}
    \item šachy\pause
    \item mariáš\pause
  \end{itemize}
  \item deterministické vs. stochastické\pause
  \begin{itemize}
    \item šachy, mariáš, vysavač\pause
    \item autonomní vozidlo\pause
  \end{itemize}
  \item epizodické vs. sekvenční\pause
  \begin{itemize}
   \item epizodické --- následující vjemy nezávislé na předchozích vjemech a akcích\pause\ (př. výstupní kontrola kvality)\pause
  \end{itemize}
  \item statické vs. dynamické\pause
  \begin{itemize}
   \item mariáš, šachy (bez hodin)\pause
   \item autonomní vozidlo, šachy
  \end{itemize}
\end{itemize}
\end{frame}
\begin{frame}\frametitle{Klasifikace Prostředí}
\begin{itemize}
  \item diskrétní vs. spojité\pause
  \begin{itemize}
   \item šachy, mariáš\pause
   \item autonomní vozidlo
  \end{itemize}\pause
  (Pozor, počítače vždy pracují s \alert{digitálním}, t.j. diskrétním vstupem; je otázka, zda svět samotný není diskrétní.)\pause
  \item jedno agentní vs. multiagentní\pause
  \begin{itemize}
   \item vysavač, šachy\pause
   \item mariáš, autonomní vozidlo
  \end{itemize}
 \end{itemize}
\end{frame}

\begin{frame}\frametitle{Racionalita}
\begin{block}{}
Racionální agent se snaží maximalizovat hodnotící funkci.
\end{block}\pause
Chceme hodnotit chování\pause nebo stav prostředí?
\begin{itemize}
 \item dobrý vysavač vysaje hodně špíny, hodnotící funkce dává body za vysátou špínu\pause
 \item[] ideální strategií je vysát špínu, pak ji vysypat, pak zase vysát a tak dokola$\ldots$\pause
 \item dobrý vysavač udržuje místnost čistou
\end{itemize}\pause

\begin{block}{}
 Obecně je lepší hodnotit stav prostředí, často totiž není jasné, jaké chování je dobré.
\end{block}
\end{frame}

\begin{frame}\frametitle{Racionalita --- varování}
Racionální neznamená\pause
\begin{itemize}
 \item vševědoucí --- vjemy nemusí poskytnout všechny relevantní informace\pause
 \item vidící do budoucnosti --- nemůže předvídat výsledky některých akcí\pause
\end{itemize}
Tudíž racionální $\neq$ úspěšný.
\end{frame}


\begin{frame}\frametitle{Základní typy agentů}
 \begin{itemize}
  \item reflexivní agenti\pause
  \begin{itemize}
   \item reagují pouze na aktuální vjem\pause
   \item pokud vjemy neposkytují dostatečný obraz světa,
         agent selže\pause
   \item částečně lze řešit randomizací akcí\pause
  \end{itemize}
  \item reflexivní agenti s vnitřním stavem\pause
  \begin{itemize}
   \item buduje si model vnějšího světa\pause
   \item učí se, jak jeho akce ovlivňují svět\pause
  \end{itemize}
  \item \uv{goal-based} agenti\pause
  \begin{itemize}
  \item agent uvažuje o budoucnosti\pause
  \item hledá posloupnost akcí, které vedou k dosažení cíle ({\bf plánování})\pause
  \end{itemize}
  \item \uv{utility-based} agenti\pause
  \begin{itemize}
   \item cíle jsou pouze binární\pause
   \item cíle jsou často konfliktní\pause
   \item agent se proto nesnaží dosáhnout cíle, nýbrž maximalizovat
         ``žádoucnost'' stavu
  \end{itemize}
 \end{itemize}
\end{frame}
\end{document}



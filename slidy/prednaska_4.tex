\pdfoutput=1
\documentclass[red,handout,professionalfont]{beamer}
% \documentclass[red,professionalfont]{beamer}
\usepackage{multimedia}
\usepackage{url}
\usepackage{amssymb}  % the check symbol 
% Python listing setup

\usepackage{color}
\usepackage[procnames]{listings}
\usepackage{textcomp}
\usepackage{setspace}
\usepackage{palatino}
\renewcommand{\lstlistlistingname}{Code Listings}
\renewcommand{\lstlistingname}{Code Listing}
\definecolor{gray}{gray}{0.5}
\definecolor{green}{rgb}{0,0.5,0}
\definecolor{lightgreen}{rgb}{0,0.7,0}
\definecolor{purple}{rgb}{0.5,0,0.5}
\definecolor{darkred}{rgb}{0.5,0,0}
\lstnewenvironment{python}[1][]{
\lstset{
language=python,
breaklines=true,
basicstyle=\ttfamily\small\setstretch{1},
stringstyle=\color{green},
showstringspaces=false,
alsoletter={1234567890},
otherkeywords={\ , \}, \{},
keywordstyle=\color{blue},
emph={access,and,as,break,class,continue,def,del,elif,else,%
except,exec,finally,for,from,global,if,import,in,is,%
lambda,not,or,pass,print,raise,return,try,while,assert},
emphstyle=\color{orange}\bfseries,
emph={[2]self},
emphstyle=[2]\color{gray},
emph={[4]ArithmeticError,AssertionError,AttributeError,BaseException,%
DeprecationWarning,EOFError,Ellipsis,EnvironmentError,Exception,%
False,FloatingPointError,FutureWarning,GeneratorExit,IOError,%
ImportError,ImportWarning,IndentationError,IndexError,KeyError,%
KeyboardInterrupt,LookupError,MemoryError,NameError,None,%
NotImplemented,NotImplementedError,OSError,OverflowError,%
PendingDeprecationWarning,ReferenceError,RuntimeError,RuntimeWarning,%
StandardError,StopIteration,SyntaxError,SyntaxWarning,SystemError,%
SystemExit,TabError,True,TypeError,UnboundLocalError,UnicodeDecodeError,%
UnicodeEncodeError,UnicodeError,UnicodeTranslateError,UnicodeWarning,%
UserWarning,ValueError,Warning,ZeroDivisionError,abs,all,any,apply,%
basestring,bool,buffer,callable,chr,classmethod,cmp,coerce,compile,%
complex,copyright,credits,delattr,dict,dir,divmod,enumerate,eval,%
execfile,exit,file,filter,float,frozenset,getattr,globals,hasattr,%
hash,help,hex,id,input,int,intern,isinstance,issubclass,iter,len,%
license,list,locals,long,map,max,min,object,oct,open,ord,pow,property,%
quit,range,raw_input,reduce,reload,repr,reversed,round,set,setattr,%
slice,sorted,staticmethod,str,sum,super,tuple,type,unichr,unicode,%
vars,xrange,zip},
emphstyle=[4]\color{purple}\bfseries,
upquote=true,
morecomment=[s][\color{lightgreen}]{"""}{"""},
commentstyle=\color{red}\slshape,
literate={>>>}{\textbf{\textcolor{darkred}{>{>}>}}}3%
         {...}{{\textcolor{gray}{...}}}3,
procnamekeys={def,class},
procnamestyle=\color{blue}\textbf,
framexleftmargin=1mm, framextopmargin=1mm,% frame=shadowbox,
%rulesepcolor=\color{blue},
#1
}}{}


\usepackage{tikz}
\usepackage{tikz-qtree}
\usetikzlibrary{decorations.pathreplacing,positioning}
%\bibliography{mujstyl}
\theoremstyle{definition}
\newtheorem{definice}{Definition}[section]
\newtheorem{thm}{Theorem}[section]
\newtheorem{idea}{Idea}[section]
\newtheorem{cor}[thm]{Corrollary}
\newtheorem{lem}[thm]{Lemma}
\newtheorem{obs}[thm]{Observation}
\newtheorem{rem}[thm]{Remark}
\newtheorem{ex}[thm]{Example}
\newtheorem{quizz}[thm]{Question}
\newcommand{\pomega}{\mbox{$\mathcal{P}(\omega)$}}
\newcommand{\cont}{\mbox{$\mathfrak c$}}
\newcommand{\ba}{\mbox{${\mathbb B}$}}
\newcommand{\0}{\mbox{${\bf 0}$}}
\newcommand{\F}{\mbox{${\mathcal F}$}}
\newcommand{\rest}{\mbox{$\upharpoonright$}}
\newcommand{\cl}[1]{\mbox{$\overline{#1}$}}
\newcommand{\yes}{\textcolor{green}{$\checkmark$}}
\newcommand{\no}{\textcolor{red}{$\times$}}
\renewcommand{\emph}[1]{{\bf #1}}
\mode<presentation>
{
\useinnertheme{rounded}

\usecolortheme{whale}
\usecolortheme{orchid}

\setbeamerfont{block title}{size={}}

%   \useoutertheme{default}
%   \usetheme{Copenhagen}
%   \useoutertheme{default}
  \setbeamercovered{invisible}
}
\setbeamertemplate{navigation symbols}{} 
\usepackage[utf8]{inputenc}
\usepackage[czech,english]{babel}
\usepackage{lmodern}
%\usepackage{times}
\usepackage[T1]{fontenc}


\tikzset{onslide/.code args={<#1>#2}{%
  \only<#1>{\pgfkeysalso{#2}} % \pgfkeysalso doesn't change the path
}}

\tikzstyle{hilight}=[red,ultra thick]
\tikzstyle{active}=[yellow,ultra thick]
\tikzstyle{memory}=[blue]



\title[]{Informovaný Agent "Hledač" \\ (4. přednáška)}

% Dnes se podíváme na jednoduché "goal-based" agenty
% a ukážeme si obecné postupy 


% \author[]{Jonathan L. Verner}
% \institute[Charles University, Prague] % (optional,but mostly needed)
% {
%   Department of Logic\\
%   Faculty of Philosophy\\
%   Charles University in Prague
% }
\date[]{}
% \subject{}
%\pgfdeclareimage[height=1cm]{university-logo}{UK-logo}
%\logo{\pgfuseimage{university-logo}}

\begin{document}

\AtBeginSection[]
{
  \begin{frame}<beamer>
    \begin{block}{}
%    \begin{centering}
    \hfill\insertsection\hfill\
 %   \end{centering}
    \end{block}
    %\tableofcontents[currentsection]
  \end{frame}
}

\AtBeginSubsection[]
{
  \begin{frame}<beamer>
    \begin{block}{}
  %  \begin{centering}
    \hfill\insertsubsection\hfill\
   % \begin{centering}
    \end{block}
    %\tableofcontents[currentsection]
  \end{frame}
}

%#################################################################

\begin{frame}{} \titlepage
%{\ \hfill \includegraphics[width=1cm]{UK-logo}\hfill\ }
\end{frame}

\begin{frame}\frametitle{Informované Prohledávání}
V minulé přednášce jsme probírali prohledávací algoritmy BFS, DFS, DLS, IDS,
které nevyužívají žádných speciálních znalostí o problému.\pause\ Problémem
je malá efektivita.\pause\ Dnes se podíváme, jak jejich efektivitu zlepšit, pokud
o konkrétním problému něco víme.
\end{frame}

\begin{frame}[fragile]\frametitle{Stromové prohledávání dle strategie --- připomenutí}
\begin{python}
def treeSearch( problem, strategy ):
    # Inicializuj strom.
    listy = node(None, problem.initial_state())
    while True:
      # Zvol kandidata k expanzi dle strategie.
      leaf_node = strategy.choose(problem,listy)
      state = leaf_node.state()
      # Mame reseni,vrat posloupnost kroku.
      if problem.is_goal( state ):
        return leaf_node
      # Expanduj kandidata (pridej do stromu 
      # uzly, ktere sousedi s kandidatem).
      for a in state.possible_actions():
        cil_stav = stav.act(a)
        child = node(leaf_node,cil_stav)
        listy.append(child)
\end{python}
\end{frame}
\begin{frame}[fragile]\frametitle{Grafové prohledávání dle strategie --- připomenutí}
\begin{python}
def treeSearch( problem, strategy ):
    # Inicializuj strom.
    listy = node(None, problem.initial_state())
    while True:
      # Zvol kandidata k expanzi dle strategie.
      leaf_node = strategy.choose(problem,listy)
      state = leaf_node.state()
      if problem.is_goal( state ):
        return leaf_node
      #GraphSearch - uklada si navstivene stavy
      visited.append( state )
      for a in state.possible_actions():
        cil_stav = state.act(a)
        if cil_stav in visited:
          continue
        child = node(leaf_node,cil_stav)
        listy.append(child)
\end{python}
\end{frame}

\begin{frame}[fragile]\frametitle{Strategie --- evaluační funkce}
\begin{python}
def strategy( problem, listy ):
    candidate = listy[0]
    fitness = evalfunc(candidate)
    for i in range(len(listy)):
      leaf = listy[i]
      if evalfunc(i,leaf) <  fitness:
          candidate = leaf
          fitness = evalfunc(leaf)
    listy.remove(candidate)
    return candidate
\end{python}\pause
Vhodnou volbou evaluační funkce {\tt evalfunc} dostáváme jednotlivé algoritmy:\pause
\begin{itemize} 
 \item BFS\pause\ --- {\tt evalfunc(i,leaf) = depth(leaf)}\pause
 \item DFS\pause\ --- {\tt evalfunc(i,leaf) = i}
\end{itemize}
\end{frame}

\begin{frame}\frametitle{Problém --- expanze neperspektivních uzlů}
\begin{block}{}
\begin{center}
 Algoritmus zbytečně expanduje neperspektivní uzly. 
\end{center}
\end{block}\pause
\begin{itemize}
 \item Hloubka uzlu (resp. jeho pořadí) nijak nesouvisí s ``perspektivností''.\pause
 \item V konkrétních případech lze často perspektivnost uzlu odhadnout:\pause
 \begin{itemize}
   \item routing\pause\ (hledání cesty na mapě)\pause\ --- perspektivní jsou ty uzly, které jsou blíž cíli\pause
   \item loydova osmička\pause\ --- perspektivní jsou ty uzly, kde je osmička blíže cílovému uspořádání\pause
 \end{itemize}
\end{itemize}
Na základě znalosti jednotlivých problémů lze nalézt vhodnout \alert{heuristickou funkci} $h(node)$\pause\ a položit např.
\begin{center}
 {\tt evalfunc(i,leaf) = h(leaf)}
\end{center}
\end{frame}

\begin{frame}\frametitle{Hladové algoritmy (greedy best-first search)}
Heuristickou funkci $h(node)$ lze interpretovat jako ``cenu'' cesty z daného uzlu do cíle.\pause\ Hladový algoritmus volí uzel s nejmenší cenou:\pause
\begin{center}
 {\tt evalfunc(i,leaf) = h(leaf)}
\end{center}\pause
\begin{itemize}
 \item úplnost:\pause\ NE (treesearch)\pause, ANO (graph search pro konečné grafy)\pause
%  problém cyklů
 \item časová složitost:\pause\ obecně $O(b^m)$\pause, dobrá heuristika může dramaticky zlepšit\pause
 \item paměťová náročnost:\pause\  obecně $O(b^m)$\pause, dobrá heuristika může dramaticky zlepšit\pause
\end{itemize}
\begin{block}{}
\begin{center}
  Momentálně nejlevnější uzel nemusí být nejvhodnější.\pause\\ (Nejsme tak bohatí, abychom si kupovali levné věci)
\end{center}
\end{block}
\end{frame}

\begin{frame}\frametitle{$A^*$ search --- minimalizace celkových nákladů}
 $A^*$ algoritmus bere v potaz nejen odhadovanou ``cenu'' cesty k cíli $h(node)$\pause, ale i cenu cesty z počátku $g(node)$.\pause

\begin{center}
 evalfunc(i,leaf) = g(leaf) + h(leaf)
\end{center}

\begin{itemize}
 \item Algoritmus tedy neprodlužuje cesty, které už jsou dlouhé.\pause
 \item Evaluační funkce udává odhadovanou cenu \alert{nejlevnější cesty} přes daný uzel.\pause
 \item Pro vhodnou volbu $h(node)$ (přípustná, konzistentní) je algoritmus optimální!
\end{itemize}
\end{frame}

\begin{frame}\frametitle{Přípustné a monotónní heuristiky}
  Heuristika je \emph{přípustná} (admisibile)\pause, pokud je vždy nižší, než minimální celková cena cesty z daného uzlu do cíle.\pause\ Heuristika
  je \emph{monotónní} (příp. konzistentní)\pause, pokud splňuje variantu tzv. trojúhelníkové nerovnosti:
  \begin{displaymath}
   h(node) \leq cost(node,action,successor) + h(successor),
  \end{displaymath}\pause
  kde $cost(node,action,successor)$ je reálná cena cesty z uzlu $node$ do následnického uzlu $successor$ pomocí akce $action$.\pause
  \vskip1cm
  \emph{Věta}: Monotónní heuristika je přípustná.
\end{frame}

\begin{frame}\frametitle{Optimalita $A^*$}
  \emph{Věta}: $A^*$ (tree search) je optimální pro přípustné heuristiky.\pause

  \medskip

  \emph{Věta}: $A^*$ (graph search) je optimální pro monotónní heuristiky.\pause

  \medskip

  Idea:\pause
  \begin{itemize}
   \item evaluační funkce je neklesající podél libovolné cesty\pause
   \item kdykoliv je zvolen uzel k expanzi, nejkratší cesta do daného uzlu je již známa
  \end{itemize}
\end{frame}

\begin{frame}\frametitle{Vlastnosti $A^*$}
 \emph{Absolutní chyba} heuristiky ($\Delta$)\pause\ je rozdíl mezi odhadovanou cenou optimálního řešení $h(root)$\pause\ a reálnou cenou optimálního řešení $h^*(root)$.\pause\
 \emph{Relativní chyba} ($\varepsilon$) je $\Delta/h^*(root)$.\pause

\begin{itemize}
 \item úplnost:\pause\ ANO (pro konečné grafy)\pause
%  problém cyklů
 \item časová složitost:\pause\ obecně $O(b^d)$\pause, resp. $O(b^{\varepsilon d})$ \pause
 \item paměťová náročnost:\pause\  obecně $O(b^d)$, resp. $O(b^{\varepsilon d})$ \pause
\end{itemize}

Největším problémem je stále \alert{paměť.}
 
\end{frame}

\begin{frame}\frametitle{$IDA^*, RBFS, SMA$}
\begin{itemize}
 \item $IDA^*$\pause\ --- podobné jako IDFS, limitem není hloubka ale hodnota evaluační funkce\pause
 \item $RBFS$\pause\ --- prohledává nejslibnější větev;\pause\ pamatuje si nejslibnější alternativu;\pause\
                         pokud cena aktuální větve přesáhne cenu alternativy\pause, aktuální větev zahodí a pokračuje v alternativě;\pause\
                         má lineární paměťovou složitost\pause
% Problém přeskakování                         
 \item $SMA$\pause\ --- lépe využívá dostupnou paměť\pause, uzly zahazuje až ve chvíli, kdy dojde paměť
\end{itemize}
\end{frame}

\begin{frame}\frametitle{Heuristiky --- kvalita, přesnost}

Kvalitu heuristiky $h$ lze měřit pomocí tzv. \emph{efektivního faktoru větvení}.\pause\ Efektivní faktor větvení $b(h)$ je nejmenší $b$ takové\pause, že
algoritmus $A^*$ s danou heuristikou expanduje tolik uzlů\pause, kolik má plný $b$-ární strom výšky $d$\pause, kde $d$ je hloubka optimálního řešení \pause.

\begin{itemize}
 \item $h(b)$ se bude lišit problém od problému\pause, pro dostatečně těžkou třídu problémů je relativně konstantní\pause
 \item v důsledku předchozího bodu lze meřit experimentálně\pause
 \item je-li $h_1\geq h_2$\pause\ (t.j. pro každý uzel $n$ platí $h_1(n)\geq h_2(n)$)\pause, pak $b(h_1)\leq b(h_2)$\pause
\end{itemize}

Máme-li dvě (přípustné, resp. konzistentní) heuristiky $h_1,h_2$\pause, lze získat novou (přípustnou, resp. konzistentní) heuristiku pomocí
\begin{displaymath}
 h(node) = \max\{h_1(node),h_2(node)\}
\end{displaymath}


\end{frame}

\begin{frame}\frametitle{Generování Heuristik --- relaxace}
% Loydova osmička:
%   kostku A lze přesunout na 
\pause
\begin{itemize}
 \item řešení problému musí splňovat nějaké podmínky\pause
 \item uvolněním podmínek získáme typicky jednodušší problém\pause
 \item řešení jednoduššího problému může být triviální\pause
 \item takto lze generovat heuristiku --- cena řešení ``uvolněného'' problému\pause
\end{itemize}
\emph{Příklad.} Loydova osmička: kostku lze přemístit z políčka $A$ na políčko $B$ pokud\pause
\begin{itemize}
  \item políčka spolu sousedí\pause\ a
  \item políčko $B$ je prázdné\pause
\end{itemize}
Relaxací těchto podmínek získáme tři různé problémy (každý jedoduše řešitelný):\pause\
Kostku lze přemístit z políčka $A$ na políčko $B$ pokud\pause
\begin{itemize}
  \item políčka spolu sousedí\pause\ (manhattan metric)\pause\ nebo
  \item políčko $B$ je prázdné\pause\ nebo
  \item kdykoliv\pause\ (počet špatně umístěných políček)
\end{itemize}
\end{frame}
\begin{frame}\frametitle{Automatické Generování Heuristik pomocí relaxace}
\begin{itemize}
 \item popsaný postup lze zformalizovat a automatizovat\pause
 \item program ABSOLVER\pause\ (první rozumná heuristika pro Rubikovu kostku, nejlepší heuristika pro Loydovu osmičku)
\end{itemize}
\end{frame}

\begin{frame}\frametitle{Generování Heuristik --- databáze vzorů}
\begin{itemize}
 \item nalezneme cenu optimálního řešení pro všechny malé ``podproblémy''\pause
 \item cena daného problému je maximum z cen těch podproblémů, které jsou v něm obsaženy\pause
 \item obecně cenu nelze sčítat\pause\ v pokud volíme podproblémy opatrně, může se to podařit
\end{itemize}
\end{frame}

\begin{frame}\frametitle{Generování Heuristik --- machine learning}
\begin{itemize}
 \item zvolíme charakteristiky problému $x_1(p),\ldots,x_n(p)$\pause\ (angl. features)\pause
 \item vyřešíme mnoho náhodně generovaných problémů $p_1,\ldots,p_N,\ N\approx 10000$\pause\ získáme tím přesnou cenu $c(p_1),\ldots,c(p_N)$
 \item snažíme se najít nejlepší lineární (případně polynomiální) funkci ``interpolující'' získaná data\pause\ t.j.
\end{itemize}

hledáme $c_1,k_1,\ldots,c_n,k_n$ tak,\pause abychom minimalizovali
\begin{displaymath}
\sum_{i=0}^N c(p_i) - \left(\sum_{j=0}^n c_j x_j(p)^{k_j}\right).
\end{displaymath}\pause

Získáme tak heuristiku:\pause

\begin{displaymath}
h(p) = \sum_{j=0}^n c_j x_j(p)^{k_j}.
\end{displaymath}\pause

\end{frame}



\end{document}



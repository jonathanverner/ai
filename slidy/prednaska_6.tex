\pdfoutput=1
% \documentclass[red,handout,professionalfont]{beamer}
\documentclass[red,professionalfont]{beamer}
\usepackage{multimedia}
\usepackage{chessboard}
\usepackage{url}
\usepackage{amssymb}  % the check symbol 
% Python listing setup

\usepackage{color}
\usepackage[procnames]{listings}
\usepackage{textcomp}
\usepackage{setspace}
\usepackage{palatino}
\renewcommand{\lstlistlistingname}{Code Listings}
\renewcommand{\lstlistingname}{Code Listing}
\definecolor{gray}{gray}{0.5}
\definecolor{green}{rgb}{0,0.5,0}
\definecolor{lightgreen}{rgb}{0,0.7,0}
\definecolor{purple}{rgb}{0.5,0,0.5}
\definecolor{darkred}{rgb}{0.5,0,0}
\lstnewenvironment{python}[1][]{
\lstset{
language=python,
breaklines=true,
basicstyle=\ttfamily\small\setstretch{1},
stringstyle=\color{green},
showstringspaces=false,
alsoletter={1234567890},
otherkeywords={\ , \}, \{},
keywordstyle=\color{blue},
emph={access,and,as,break,class,continue,def,del,elif,else,%
except,exec,finally,for,from,global,if,import,in,is,%
lambda,not,or,pass,print,raise,return,try,while,assert},
emphstyle=\color{orange}\bfseries,
emph={[2]self},
emphstyle=[2]\color{gray},
emph={[4]ArithmeticError,AssertionError,AttributeError,BaseException,%
DeprecationWarning,EOFError,Ellipsis,EnvironmentError,Exception,%
False,FloatingPointError,FutureWarning,GeneratorExit,IOError,%
ImportError,ImportWarning,IndentationError,IndexError,KeyError,%
KeyboardInterrupt,LookupError,MemoryError,NameError,None,%
NotImplemented,NotImplementedError,OSError,OverflowError,%
PendingDeprecationWarning,ReferenceError,RuntimeError,RuntimeWarning,%
StandardError,StopIteration,SyntaxError,SyntaxWarning,SystemError,%
SystemExit,TabError,True,TypeError,UnboundLocalError,UnicodeDecodeError,%
UnicodeEncodeError,UnicodeError,UnicodeTranslateError,UnicodeWarning,%
UserWarning,ValueError,Warning,ZeroDivisionError,abs,all,any,apply,%
basestring,bool,buffer,callable,chr,classmethod,cmp,coerce,compile,%
complex,copyright,credits,delattr,dict,dir,divmod,enumerate,eval,%
execfile,exit,file,filter,float,frozenset,getattr,globals,hasattr,%
hash,help,hex,id,input,int,intern,isinstance,issubclass,iter,len,%
license,list,locals,long,map,max,min,object,oct,open,ord,pow,property,%
quit,range,raw_input,reduce,reload,repr,reversed,round,set,setattr,%
slice,sorted,staticmethod,str,sum,super,tuple,type,unichr,unicode,%
vars,xrange,zip},
emphstyle=[4]\color{purple}\bfseries,
upquote=true,
morecomment=[s][\color{lightgreen}]{"""}{"""},
commentstyle=\color{red}\slshape,
literate={>>>}{\textbf{\textcolor{darkred}{>{>}>}}}3%
         {...}{{\textcolor{gray}{...}}}3,
procnamekeys={def,class},
procnamestyle=\color{blue}\textbf,
framexleftmargin=1mm, framextopmargin=1mm,% frame=shadowbox,
%rulesepcolor=\color{blue},
#1
}}{}


\usepackage{tikz}
\usepackage{tikz-qtree}
\usetikzlibrary{decorations.pathreplacing,positioning}
%\bibliography{mujstyl}
\theoremstyle{definition}
\newtheorem{definice}{Definition}[section]
\newtheorem{thm}{Theorem}[section]
\newtheorem{idea}{Idea}[section]
\newtheorem{cor}[thm]{Corrollary}
\newtheorem{lem}[thm]{Lemma}
\newtheorem{obs}[thm]{Observation}
\newtheorem{rem}[thm]{Remark}
\newtheorem{ex}[thm]{Example}
\newtheorem{quizz}[thm]{Question}
\newcommand{\pomega}{\mbox{$\mathcal{P}(\omega)$}}
\newcommand{\cont}{\mbox{$\mathfrak c$}}
\newcommand{\ba}{\mbox{${\mathbb B}$}}
\newcommand{\0}{\mbox{${\bf 0}$}}
\newcommand{\F}{\mbox{${\mathcal F}$}}
\newcommand{\rest}{\mbox{$\upharpoonright$}}
\newcommand{\cl}[1]{\mbox{$\overline{#1}$}}
\newcommand{\yes}{\textcolor{green}{$\checkmark$}}
\newcommand{\no}{\textcolor{red}{$\times$}}
\renewcommand{\emph}[1]{{\bf #1}}
\mode<presentation>
{
\useinnertheme{rounded}

\usecolortheme{whale}
\usecolortheme{orchid}

\setbeamerfont{block title}{size={}}

%   \useoutertheme{default}
%   \usetheme{Copenhagen}
%   \useoutertheme{default}
  \setbeamercovered{invisible}
}
\setbeamertemplate{navigation symbols}{} 
\usepackage[utf8]{inputenc}
\usepackage[czech,english]{babel}
\usepackage{lmodern}
%\usepackage{times}
\usepackage[T1]{fontenc}


\tikzset{onslide/.code args={<#1>#2}{%
  \only<#1>{\pgfkeysalso{#2}} % \pgfkeysalso doesn't change the path
}}

\tikzstyle{hilight}=[red,ultra thick]
\tikzstyle{active}=[yellow,ultra thick]
\tikzstyle{memory}=[blue]



\title[]{CSP \\ (6. přednáška)}

% Dnes se podíváme na jednoduché "goal-based" agenty
% a ukážeme si obecné postupy 


% \author[]{Jonathan L. Verner}
% \institute[Charles University, Prague] % (optional,but mostly needed)
% {
%   Department of Logic\\
  %   Faculty of Philosophy\\
%   Charles University in Prague
% }
\date[]{}
% \subject{}
%\pgfdeclareimage[height=1cm]{university-logo}{UK-logo}
%\logo{\pgfuseimage{university-logo}}

\begin{document}

\AtBeginSection[]
{
  \begin{frame}<beamer>
    \begin{block}{}
%    \begin{centering}
    \hfill\insertsection\hfill\
 %   \end{centering}
    \end{block}
    %\tableofcontents[currentsection]
  \end{frame}
}

\AtBeginSubsection[]
{
  \begin{frame}<beamer>
    \begin{block}{}
  %  \begin{centering}
    \hfill\insertsubsection\hfill\
   % \begin{centering}
    \end{block}
    %\tableofcontents[currentsection]
  \end{frame}
}

%#################################################################

\begin{frame}{} \titlepage
%{\ \hfill \includegraphics[width=1cm]{UK-logo}\hfill\ }
\end{frame}

\begin{frame}\frametitle{Kdo chová zebru?}

{\tiny
\begin{itemize}
\item[1.] There are five houses.
\item[2.] The Englishman lives in the red house.
\item[3.] The Spaniard owns the dog.
\item[4.] Coffee is drunk in the green house.
\item[5.] The Ukrainian drinks tea.
\item[6.] The green house is immediately to the right of the ivory house.
\item[7.] The Old Gold smoker owns snails.
\item[8.] Kools are smoked in the yellow house.
\item[9.] Milk is drunk in the middle house.
\item[10.] The Norwegian lives in the first house.
\item[11.] The man who smokes Chesterfields lives in the house next to the man with the fox.
\item[12.] Kools are smoked in a house next to the house where the horse is kept.
\item[13.] The Lucky Strike smoker drinks orange juice.
\item[14.] The Japanese smokes Parliaments.
\item[15.] The Norwegian lives next to the blue house. 
\end{itemize}\pause

Now, who drinks water? Who owns the zebra? In the interest of clarity, it must be added that each of the five houses is painted a different color, and their inhabitants are of different national extractions, own different pets, drink different beverages and smoke different brands of American cigarets [sic]. One other thing: in statement 6, right means your right.

{\it \hfill — Life International, December 17, 1962}

}

\end{frame}

\begin{frame}\frametitle{CSP --- Constraint Satisfaction Problem}
\begin{description}
 \item[$X_0,\ldots X_n$] proměnné\pause
 \item[$D_0,\ldots,D_n$] obory hodnot jednotlivých proměnných\pause
 \item[$C_0,\ldots,C_k$] omezující podmínky,\pause\ $C_i = (A_i,R_i)$\pause\ kde
 \item[$A_i$] $=\{j_0,\ldots,j_{l_i}\}\subseteq n+1$,\pause\ seznam proměnných vstupujících do této podmínky\pause
 \item[$R_i$] $\subseteq D_{j_0}\times\cdots\times D_{j_{l_i}}$\pause\  povolené hodnoty proměnných\pause
\end{description}
\begin{center}
 CSP
\end{center}\pause
% \begin{block}{}
Nalezněte ohodnocení proměnných $x_0,\ldots,x_n$\pause\ hodnotami $v_0,\ldots,v_n$ z odpovídajích oborů hodnot\pause\ splňující
všechny omezující podmínky,\pause\ t.j. $(v_{j_0},\ldots,v_{j_{l_i}})\in R_i$ pro každé $i=0,\ldots,k$.
% \end{block}
\end{frame}

\begin{frame}\frametitle{Zebra formulovaná jako CSP}
\begin{center}
Proměnné
\end{center}\pause
yellow, blue, red, ivory, green, norwegian, ukrainian, englishman, spaniard, japanese, water, tea, milk, orange juice, coffee, 
kools, chesterfield, old gold, lucky strike, parliament fox,  horse, snails, dog, zebra\pause
\begin{center}
Obory hodnot\pause
\vskip 2mm
$\{1,2,3,4,5\}$

{\tiny (odpovídají jednotlivým domům)}

\end{center}\pause

\begin{center}
 Omezující podmínky\pause
 \vskip2mm
{\tiny
\begin{minipage}{5cm}
\begin{itemize}
\item[2.]<6-> The Englishman lives in the red house.
\item[3.]<8-> The Spaniard owns the dog.
\item[$\ldots$]<10->\ 
\item[15.]<11-> The Norwegian lives next to the blue house. 
\item[A]<13-> it must be added that each of the five houses is painted a different color
\item[$\ldots$]<15->\ 
\end{itemize}\pause
\end{minipage}
\begin{minipage}{5.2cm}
\begin{itemize}
 \item[2.]<7-> englishman = red
 \item[3.]<9-> spaniard = dog
 \item[$\ldots$]<10->\ 
 \item[15.]<12-> norwegian = blue + 1 or norwegian = blue - 1
 \item[A]<14-> proměnné yellow, blue, red, ivory, green mají každá jinou hodnotu
 \item[$\ldots$]<15->\ 
\end{itemize}
\end{minipage}
}
\end{center}

\end{frame}

\begin{frame}\frametitle{Další CSP problémy}
 \begin{itemize}
  \item[] Sudoku\pause
  \item[] obarvení grafu\pause
  \item[] umístění dam na šachovnici\pause
  \item[] křížovky\pause
  \item[] \emph{SAT}\pause
  \item[] BID (building identification problem)\pause
  \item[] rozvrhování, plánování $\ldots$
 \end{itemize}
\end{frame}


\begin{frame}[fragile]\frametitle{CSP Solver: Backtracking}
\begin{python}[basicstyle=\ttfamily\tiny\setstretch{1}]
def CSPBackTrackSolver(problem, partial_assignment):
  var = selectUnassignedVar(problem, partial_assignment)
  
  # pokud jsou vsechny promenne prirazene, uspeli jsme
  if var is None:
    return partial_assignment

  for val in problem.domain(var):
      partial_assignment[var] = val
      
      # pokud jsme stale konzistentni, rekurzivne hledame dal
      if problem.isConsistent(partial_assignment):
        ret = CSPBackTrackSolver(problem, partial_assignment)
        
        # hura, nasli jsme reseni
        if ret:
          return ret
  return None
\end{python}
\end{frame}

\begin{frame}[fragile]\frametitle{CSP Solver: Zlepšení}
 \begin{itemize}
  \item vhodně volit pořadí přiřazování proměnných ({\tt selectUnassignedVar})\pause
  \item vhodně volit pořadí hodnot\pause
  \item neprocházet neperspektivní hodnoty (\emph{constraint propagation})\pause
  \item vynechávat podstromy, o kterých víme, že jsou neperspektivní (\emph{backjumping})\pause
 \end{itemize}
\end{frame}

\begin{frame}[fragile]\frametitle{Pořadí proměnných --- {\tt selectUnassignedVar}}
\begin{block}{}
Zvolme proměnnou, která má nejméně možných přiřazení konzistentních s aktuálním přiřazením.
\end{block}\pause
\begin{itemize}
 \item spolu s v přiřazení si budeme pro proměnné pamatovat konzistentní hodnoty: {\tt pa.restricted} ---
       seznam dvojic (proměnná, množina legálních hodnot)\pause
 \item při každém přiřazení je třeba aktualizovat {\tt pa.restricted}
\end{itemize}\pause

\begin{python}[basicstyle=\ttfamily\tiny\setstretch{1}]
def selectUnassignedVar(problem, partial_assignment):
  if len(partial_assignment.restricted()) > 0:
    var, dom = min(partial_assignment.restricted, 
                   key = lambda (var,dom): len(dom))
    return var
  else:
    for var in problem.vars:
      if var not in partial_assignment:
        return var
  return None
\end{python}

\end{frame}

\begin{frame}[fragile]\frametitle{Constraint propagation}
\begin{itemize}
 \item založeno na pojmu tzv. 2-konzistence\pause
 \item pro jednoduchost předpokládejme, že všechny omezující podmínky jsou binární\pause
 \item obecný případ lze převést na binární podmínky,\pause\ za cenu zvýšení počtu proměnných\pause\ (cvičení)\pause
\end{itemize}

Definujme tzv. \emph{constraint graph}
\begin{description}
 \item[vrcholy] --- jednotlivé proměnné\pause
 \item[hrany]   --- jednotlivé omezení (dvě proměnné jsou spojeny hranou, pokud figurují v nějaké omezující podmínce)
\end{description}\pause

Omezené obory hodnot proměnných jsou \emph{2-konzistentní}, pokud je každé částečné přiřazení definované na
dvou proměnných\pause\ (a mající hodnoty v těchto omezených oborech)\pause\ konzistentní.

\end{frame}

\begin{frame}[fragile]\frametitle{Constraint propagation --- implementace}

\begin{python}[basicstyle=\ttfamily\tiny\setstretch{1}] 
 def assignAndReduce(problem, partial_assignment, var,val):
 """ Rozsiri partial_assignment o var=val a aktualizuje 
     partial_assignment.restricted """
  import copy
  ret_assignment = copy.deepcopy(partial_assignment)
  ret_assignment[var]=val
  work = []
  for cvar in problem.constrainedby(var):
    work.append((var,cvar))

  while len(work) > 0:
    (vA,vB) = work.pop()
    if vA in ret_assignment:
      restricted, consistent = restrict(problem, ret_assignment, vA, vB)
      if not consistent:
        return None
      if restricted:
        for cvar in problem.constrainedby(vB):
        if not cvar == vA:
          work.append((vB,cvar))
  return ret_assignment
\end{python}

\end{frame}

\begin{frame}[fragile]\frametitle{Constraint propagation --- implementace (pokračování)}
\begin{python}[basicstyle=\ttfamily\tiny\setstretch{1}] 
 def restrict(problem, ret_assignment, vA, vB):
 """ Aktualizuje partial_assignment.restricted pro promennou vB
     na zaklade omezeneho oboru hodnot promenne vA."""
\end{python}\pause
\begin{python}[basicstyle=\ttfamily\tiny\setstretch{1}]     
  if varB not in assignment.restricted:
    dom = assignment.restricted[varB] = set(problem.domain(varB))
  else:
    dom = assignment.restricted[varB]
\end{python}\pause
\begin{python}[basicstyle=\ttfamily\tiny\setstretch{1}]  
  # Hodnoty, ktere budeme zahazovat
  removed = set([])
  for valB in dom:
    # Pokud neexistuje zadne legalni prirazeni promenné A tak,
    # aby nejaka omezujici podminka na A a B nezakazovala hodnotu valB
    # tak hodnotu valB vyhodime
    if not isLegal(problem, assignment, varA, varB, valB):
      removed.add(valB)
\end{python}\pause
\begin{python}[basicstyle=\ttfamily\tiny\setstretch{1}]  
  # Obor hodnot nebyl nijak omezen, nenasli jsme nekonzistenci
  if len(removed) == 0:
    restricted=False
 
  # Aktualizuj assignment.restricted pro proměnnou varB
  else:
    restricte = True
    dom.difference_update(removed)
\end{python}\pause
\begin{python}[basicstyle=\ttfamily\tiny\setstretch{1}]  
  # Pokud nezbyly zadne hodnoty, jsme nekonzistentni
  if len(dom) == 0:
    consistent = False
    raise cspError.EmptyDomainInAssignment
  else:
    consistent = True
  return restricted,consistent
\end{python}
\end{frame}


\begin{frame}\frametitle{CSP --- složitost}
\begin{itemize}
 \item mají-li proměnné spojitý obor hodnot a podmínky jsou lineární\pause\ (resp. kvadratické)\pause,
       existuje polynomiální algoritmus\pause
 \item pokud má constraint graf speciální strukturu\pause\ (strom)\pause\ existuje polynomiální řešení\pause
 \item obecně NP-úplný problém\pause
 \item dokonce i když omezíme hodnoty proměnných na 0,1\pause\ (\emph{SAT})\pause, je problém NP-úplný.
\end{itemize}
\end{frame}

\begin{frame}\frametitle{SAT --- Boolean satisfiability problem}
\begin{itemize}
 \item 3SAT --- splnitelnost formule v CNF (conjunctive normal form), kde každá klauzule má nejvýše
       tři literály\pause
 \item HORNSAT, 3SAT, XORSAT, DNF-SAT --- polynomiální čas (DNF SAT dokonce lineární)\pause
\end{itemize}


% SAT --- NP-complete (3-sat: CNF with each clause at most three literals) is NP-complete)
%	  Note: algorithm deciding solvability in O(f) => algorithm for finding solution in O(nf)
%         Note: DNF --- easy, some formulas grow exponentially when converted to DNF
%	  HORNSAT.2SAT,XORSAT --- polynomial time
%	  3-sat --- randomized algorithm, correct result with high probability, runtime (4/3)**n
%	  SCHAFER dichotomy --- for each form F the CFSAT is either P or NP-complete
\end{frame}


\end{document}





\pdfoutput=1
% \documentclass[red,handout,professionalfont]{beamer}
\documentclass[red,professionalfont]{beamer}
\usepackage{multimedia}
\usepackage{chessboard}
\usepackage{url}
\usepackage{amssymb}  % the check symbol 
\include{pythonlisting}
\usepackage{tikz}
\usepackage{tikz-qtree}
\usetikzlibrary{decorations.pathreplacing,positioning}
%\bibliography{mujstyl}
\theoremstyle{definition}
\newtheorem{definice}{Definition}[section]
\newtheorem{thm}{Theorem}[section]
\newtheorem{idea}{Idea}[section]
\newtheorem{cor}[thm]{Corrollary}
\newtheorem{lem}[thm]{Lemma}
\newtheorem{obs}[thm]{Observation}
\newtheorem{rem}[thm]{Remark}
\newtheorem{ex}[thm]{Example}
\newtheorem{quizz}[thm]{Question}
\newcommand{\pomega}{\mbox{$\mathcal{P}(\omega)$}}
\newcommand{\cont}{\mbox{$\mathfrak c$}}
\newcommand{\ba}{\mbox{${\mathbb B}$}}
\newcommand{\0}{\mbox{${\bf 0}$}}
\newcommand{\F}{\mbox{${\mathcal F}$}}
\newcommand{\rest}{\mbox{$\upharpoonright$}}
\newcommand{\cl}[1]{\mbox{$\overline{#1}$}}
\newcommand{\yes}{\textcolor{green}{$\checkmark$}}
\newcommand{\no}{\textcolor{red}{$\times$}}
\renewcommand{\emph}[1]{{\bf #1}}
\mode<presentation>
{
\useinnertheme{rounded}

\usecolortheme{whale}
\usecolortheme{orchid}

\setbeamerfont{block title}{size={}}

%   \useoutertheme{default}
%   \usetheme{Copenhagen}
%   \useoutertheme{default}
  \setbeamercovered{invisible}
}
\setbeamertemplate{navigation symbols}{} 
\usepackage[utf8]{inputenc}
\usepackage[czech,english]{babel}
\usepackage{lmodern}
%\usepackage{times}
\usepackage[T1]{fontenc}


\tikzset{onslide/.code args={<#1>#2}{%
  \only<#1>{\pgfkeysalso{#2}} % \pgfkeysalso doesn't change the path
}}

\tikzstyle{hilight}=[red,ultra thick]
\tikzstyle{active}=[yellow,ultra thick]
\tikzstyle{memory}=[blue]



\title[]{Znalostní agenti II.\\ (8. přednáška)}

% Dnes se podíváme na jednoduché "goal-based" agenty
% a ukážeme si obecné postupy 


% \author[]{Jonathan L. Verner}
% \institute[Charles University, Prague] % (optional,but mostly needed)
% {
%   Department of Logic\\
  %   Faculty of Philosophy\\
%   Charles University in Prague
% }
\date[]{}
% \subject{}
%\pgfdeclareimage[height=1cm]{university-logo}{UK-logo}
%\logo{\pgfuseimage{university-logo}}

\begin{document}

\AtBeginSection[]
{
  \begin{frame}<beamer>
    \begin{block}{}
%    \begin{centering}
    \hfill\insertsection\hfill\
 %   \end{centering}
    \end{block}
    %\tableofcontents[currentsection]
  \end{frame}
}

\AtBeginSubsection[]
{
  \begin{frame}<beamer>
    \begin{block}{}
  %  \begin{centering}
    \hfill\insertsubsection\hfill\
   % \begin{centering}
    \end{block}
    %\tableofcontents[currentsection]
  \end{frame}
}

%#################################################################

\begin{frame}{} \titlepage
%{\ \hfill \includegraphics[width=1cm]{UK-logo}\hfill\ }
\end{frame}


% 
% \begin{frame}\frametitle{CSP --- NP-úplnost}
% \end{frame}

\begin{frame}\frametitle{Omezenost výrokové logiky}
\begin{center}
 Výroková logika je omezená
\end{center}
\end{frame}

\begin{frame}\frametitle{Predikátová logika (1. řádu)}
\begin{itemize}
 \item[-] logické spojky  ($\wedge,\vee,\rightarrow$)
 \item[-] kvantifikátory ($\exists,\forall$)
 \item[-] predikáty ($P,Q,R\ldots$)
 \item[-] funkční symboly ($F,G\ldots$)
 \item[-] konstanty ($c_0,c_1,\ldots$)
 \item[-] proměnné ($x_0,x_1,\ldots$)
\end{itemize}\pause
Víme co je to term\pause, atomická formule\pause, formule\pause, volná a vázaná proměnná\pause, $\ldots$\pause
\begin{block}{}
\begin{center}
Ground term
\end{center}
\end{block}\pause
Term $t$ je \emph{uzeměný} (ground term) pokud se v něm nevyskytují proměnné.
\end{frame}

\begin{frame}\frametitle{Převod na prvořádovou logiku}
\begin{block}{}
 \begin{center}
  Idea
 \end{center}
\end{block}\pause
\begin{itemize}
 \item[1.] Zbav se existenčních kvantorů skolemizací\pause
 \item[2.] Univerzálních kvantorů se zbav aplikováním všech možných substitucí:
\end{itemize}\pause
Je-li $t$ uzeměný term pak\pause
\begin{displaymath}
  \frac{(\forall x)\varphi}{\varphi[x/t]}
\end{displaymath}\pause
\begin{itemize}
 \item[3.] Výsledkem je výroková formule, aplikuj rezoluci
\end{itemize}\pause
\begin{block}{}
 \begin{center}
  Problém
 \end{center}
\end{block}\pause
Krok č. 2 generuje příliš mnoho formulí, často je většina naprosto nepotřebná.
\end{frame}
\begin{frame}\frametitle{Příklad}
Víme, že:\pause
\begin{itemize}
 \item[] $(\forall x)(King(x)\wedge Greedy(x)\rightarrow Evil(x))$\pause
 \item[] $King(John)$\pause
 \item[] $Brother(Richard,John)$\pause
 \item[] $Peasant(David)$\pause
 \item[] $(\forall y)Geedy(y)$\pause
\end{itemize}
Chceme zjistit, zda\pause
\begin{displaymath}
 Evil(John)
\end{displaymath}\pause
Předchozí postup generuje následující výrokové formule:\pause
\begin{center}
 $Greedy(David)$\pause, 
 $Greedy(Richard)$\pause, 
 \alt<20->{\alert{$Greedy(John)$}}{$Greedy(John)$}\pause, 
 \alt<20->{\alert{$King(John)$}}{$King(John)$}\pause,
 $Peasant(David)$\pause,
 $Brother(Richard,John)$\pause,
 $King(David)\wedge Greedy(David) \rightarrow Evil(David)$\pause,
 $King(Richard)\wedge Greedy(Richard) \rightarrow Evil(Richard)$\pause,
 \alt<20->{\alert{$King(John)\wedge Greedy(John) \rightarrow Evil(John)$}}{$King(John)\wedge Greedy(John) \rightarrow Evil(John)$},\pause
\end{center}
ale k důkazu nám stačí pouze \alt<20->{\alert{zvýrazněné formule}}{zvýrazněné formule} $\ldots$
\end{frame}

\begin{frame}\frametitle{Generalizovaný modus ponens}
Máme-li formuli tvaru\pause
\begin{displaymath}
 (\forall\bar{x})(\varphi_1(\bar{x})\wedge\cdots\varphi_n(\bar{x})\rightarrow\varphi(\bar{x}))\pause,
\end{displaymath}
formule $\psi_0(\bar{x}),\ldots,\psi_n(\bar{x})$\pause\ a uzeměné termy $\bar{t}$ takové, že\pause
\begin{displaymath}
   \psi_i(\bar{x})[\bar{x}/\bar{t}] = \varphi_i(\bar{x})[\bar{x}/\bar{t}],\medskip i=0,\ldots,n\pause,
\end{displaymath}
pak platí
\begin{displaymath}
 \frac{(\forall \bar{x})(\varphi_1\wedge\cdots\wedge\varphi_n\rightarrow\varphi),
       \psi_0,\ldots,\psi_n}
      {\varphi[\bar{x}/t]}
\end{displaymath}
\end{frame}
\begin{frame}\frametitle{Unifikace}
 \begin{center}
  Chceme najít termy $\bar{t}$ takové, že\pause
  \begin{displaymath}
   \psi(\bar{x})[\bar{x}/\bar{t}] = \varphi(\bar{x})[\bar{x}/\bar{t}]\pause,
\end{displaymath}
 \end{center}
\end{frame}





\end{document}
\pdfoutput=1
% \documentclass[red,handout,professionalfont]{beamer}
\documentclass[red,professionalfont]{beamer}
\usepackage{multimedia}
\usepackage{chessboard}
\usepackage{url}
\usepackage{amssymb}  % the check symbol 
% Python listing setup

\usepackage{color}
\usepackage[procnames]{listings}
\usepackage{textcomp}
\usepackage{setspace}
\usepackage{palatino}
\renewcommand{\lstlistlistingname}{Code Listings}
\renewcommand{\lstlistingname}{Code Listing}
\definecolor{gray}{gray}{0.5}
\definecolor{green}{rgb}{0,0.5,0}
\definecolor{lightgreen}{rgb}{0,0.7,0}
\definecolor{purple}{rgb}{0.5,0,0.5}
\definecolor{darkred}{rgb}{0.5,0,0}
\lstnewenvironment{python}[1][]{
\lstset{
language=python,
breaklines=true,
basicstyle=\ttfamily\small\setstretch{1},
stringstyle=\color{green},
showstringspaces=false,
alsoletter={1234567890},
otherkeywords={\ , \}, \{},
keywordstyle=\color{blue},
emph={access,and,as,break,class,continue,def,del,elif,else,%
except,exec,finally,for,from,global,if,import,in,is,%
lambda,not,or,pass,print,raise,return,try,while,assert},
emphstyle=\color{orange}\bfseries,
emph={[2]self},
emphstyle=[2]\color{gray},
emph={[4]ArithmeticError,AssertionError,AttributeError,BaseException,%
DeprecationWarning,EOFError,Ellipsis,EnvironmentError,Exception,%
False,FloatingPointError,FutureWarning,GeneratorExit,IOError,%
ImportError,ImportWarning,IndentationError,IndexError,KeyError,%
KeyboardInterrupt,LookupError,MemoryError,NameError,None,%
NotImplemented,NotImplementedError,OSError,OverflowError,%
PendingDeprecationWarning,ReferenceError,RuntimeError,RuntimeWarning,%
StandardError,StopIteration,SyntaxError,SyntaxWarning,SystemError,%
SystemExit,TabError,True,TypeError,UnboundLocalError,UnicodeDecodeError,%
UnicodeEncodeError,UnicodeError,UnicodeTranslateError,UnicodeWarning,%
UserWarning,ValueError,Warning,ZeroDivisionError,abs,all,any,apply,%
basestring,bool,buffer,callable,chr,classmethod,cmp,coerce,compile,%
complex,copyright,credits,delattr,dict,dir,divmod,enumerate,eval,%
execfile,exit,file,filter,float,frozenset,getattr,globals,hasattr,%
hash,help,hex,id,input,int,intern,isinstance,issubclass,iter,len,%
license,list,locals,long,map,max,min,object,oct,open,ord,pow,property,%
quit,range,raw_input,reduce,reload,repr,reversed,round,set,setattr,%
slice,sorted,staticmethod,str,sum,super,tuple,type,unichr,unicode,%
vars,xrange,zip},
emphstyle=[4]\color{purple}\bfseries,
upquote=true,
morecomment=[s][\color{lightgreen}]{"""}{"""},
commentstyle=\color{red}\slshape,
literate={>>>}{\textbf{\textcolor{darkred}{>{>}>}}}3%
         {...}{{\textcolor{gray}{...}}}3,
procnamekeys={def,class},
procnamestyle=\color{blue}\textbf,
framexleftmargin=1mm, framextopmargin=1mm,% frame=shadowbox,
%rulesepcolor=\color{blue},
#1
}}{}


\usepackage{tikz}
\usepackage{tikz-qtree}
\usetikzlibrary{decorations.pathreplacing,positioning}
%\bibliography{mujstyl}
\theoremstyle{definition}
\newtheorem{definice}{Definition}[section]
\newtheorem{thm}{Theorem}[section]
\newtheorem{idea}{Idea}[section]
\newtheorem{cor}[thm]{Corrollary}
\newtheorem{lem}[thm]{Lemma}
\newtheorem{obs}[thm]{Observation}
\newtheorem{rem}[thm]{Remark}
\newtheorem{ex}[thm]{Example}
\newtheorem{quizz}[thm]{Question}
\newcommand{\pomega}{\mbox{$\mathcal{P}(\omega)$}}
\newcommand{\cont}{\mbox{$\mathfrak c$}}
\newcommand{\ba}{\mbox{${\mathbb B}$}}
\newcommand{\0}{\mbox{${\bf 0}$}}
\newcommand{\F}{\mbox{${\mathcal F}$}}
\newcommand{\rest}{\mbox{$\upharpoonright$}}
\newcommand{\cl}[1]{\mbox{$\overline{#1}$}}
\newcommand{\yes}{\textcolor{green}{$\checkmark$}}
\newcommand{\no}{\textcolor{red}{$\times$}}
\renewcommand{\emph}[1]{{\bf #1}}
\mode<presentation>
{
\useinnertheme{rounded}

\usecolortheme{whale}
\usecolortheme{orchid}

\setbeamerfont{block title}{size={}}

%   \useoutertheme{default}
%   \usetheme{Copenhagen}
%   \useoutertheme{default}
  \setbeamercovered{invisible}
}
\setbeamertemplate{navigation symbols}{} 
\usepackage[utf8]{inputenc}
\usepackage[czech,english]{babel}
\usepackage{lmodern}
%\usepackage{times}
\usepackage[T1]{fontenc}


\tikzset{onslide/.code args={<#1>#2}{%
  \only<#1>{\pgfkeysalso{#2}} % \pgfkeysalso doesn't change the path
}}

\tikzstyle{hilight}=[red,ultra thick]
\tikzstyle{active}=[yellow,ultra thick]
\tikzstyle{memory}=[blue]



\title[]{Prolog\\ (9. přednáška)}

% Dnes se podíváme na jednoduché "goal-based" agenty
% a ukážeme si obecné postupy 


% \author[]{Jonathan L. Verner}
% \institute[Charles University, Prague] % (optional,but mostly needed)
% {
%   Department of Logic\\
  %   Faculty of Philosophy\\
%   Charles University in Prague
% }
\date[]{}
% \subject{}
%\pgfdeclareimage[height=1cm]{university-logo}{UK-logo}
%\logo{\pgfuseimage{university-logo}}

\begin{document}

\AtBeginSection[]
{
  \begin{frame}<beamer>
    \begin{block}{}
%    \begin{centering}
    \hfill\insertsection\hfill\
 %   \end{centering}
    \end{block}
    %\tableofcontents[currentsection]
  \end{frame}
}

\AtBeginSubsection[]
{
  \begin{frame}<beamer>
    \begin{block}{}
  %  \begin{centering}
    \hfill\insertsubsection\hfill\
   % \begin{centering}
    \end{block}
    %\tableofcontents[currentsection]
  \end{frame}
}

%#################################################################

\begin{frame}{} \titlepage
%{\ \hfill \includegraphics[width=1cm]{UK-logo}\hfill\ }
\end{frame}

\begin{frame}\frametitle{Syntax prologu --- stavební kameny}
\begin{description}
 \item[predikáty] začínají malým písmenem, mohou obsahovat číslice a podtržítka\pause
 \item[funkce] začínají malým písmenem, mohou obsahovat číslice a podtržítka\pause
 \item[konstanty] (nulární predikáty)\pause
 \item[proměnné] začínají velkým písmenem nebo podtržítkem\pause

 \item[spojky] {\tt :-} (implikace)\pause, {\tt ,} (konjunkce)\pause, {\tt ;} (disjunkce)\pause, {\tt =} (rovnost)
\end{description}
\end{frame}

\begin{frame}[fragile]\frametitle{Syntax prologu --- statements}
\alert{Fakta}\pause
\begin{verbatim}
 muz(adam).
 zena(eva).
 manzel(adam,eva).
 rodic(adam,abel).
 rodic(eva,abel).
\end{verbatim}\pause
\alert{Procedury/Pravidla}\pause\ ({\tt Hlava :- Klauzule })\pause
\\
{\tt
 matka(M,D) :- rodic(M,D), zena(M).\\
 otec(O,D) :- rodic(O,D), muz(O).\pause\\
 sourozenec(X,Y) :- rodic(R,X), rodic(R,Y).\pause\\
 vlastni\_sourozenec(X,Y) :-\\
\hskip1cm       otec(O,X), otec(O,Y),\\
\hskip1cm       matka(M,X), matka(M,Y).\pause\\
}
\begin{block}{}
Prologovský program je seznam (databáze) faktů a procedur. 
\end{block}
\end{frame}

\begin{frame}[fragile]\frametitle{Interakce s prologem}
\begin{itemize}
 \item Načteme databázi faktů ({\tt consult}).\pause
 \item Klademe dotazy\pause
\end{itemize}
\begin{block}{}
 \begin{center}
  Příklad interakce
 \end{center}
\end{block}\pause
{\tt
1 ?-\pause\ consult(genealogy).\pause\\
\% genealogy compiled 0.00 sec, 10 clauses\\
true.\\
\
\\
2 ?-\pause\ rodic(adam,abel).\pause\\
true.\\
\
\\
3 ?-\pause\ rodic(X,abel).\pause\\
X = adam\pause\ ;\pause\\
X = eva.\\
\
\\
4 ?-
}
\end{frame}

\begin{frame}\frametitle{Hledání důkazu}
 \begin{itemize}
  \item unifikace\pause
  \item prohledávání do hloubky a ``backtracking''\pause
 \end{itemize}
\begin{block}{}
 \begin{center}
      {\tt sourozenec(X,abel)}
 \end{center}
\end{block}\pause

 \begin{itemize}
  \item prochází databázi a snaží se unifikovat sourozenec(X,abel) s ``hlavou'' nějakého faktu, nebo pravidla (procedury).\pause
  \item pokud se mu podaří unifikace s faktem, uspěl.\pause
  \item pokud se mu podaří unifikace s hlavou pravidla, pokusí se rekurzivně provést totéž s jednotlivými klauzulemi pravidla\pause
 \end{itemize}
{\tiny\tt
Cíl: sourozenec(X,abel)\pause\\
\hskip0.5cm  sourozenec(X,abel) = sourozenec(X,Y) / [Y=abel]\pause\\
\hskip1cm    Cíl: rodic(R,X), rodic(R, Y) / [Y=abel]\pause\\
\hskip1.5cm      rodic(R,X) = rodic(adam,kain) / [Y=abel,R=adam,X=kain]\pause\\
\hskip2cm        Cíl: rodic(R,Y) / [Y=abel,R=adam,X=kain]\pause\\
\hskip2.5cm           rodic(R,Y) = rodic(adam,abel) / [Y=abel,R=adam,X=kain]\pause\\
\hskip2.5cm           \alert{ÚSPĚCH}\pause\\
}
\begin{itemize}
 \item chtěli bychom, aby neplatilo {\tt sourozenec(abel,abel)};\pause\ to však nelze zapsat hornovskou klauzulí;\pause\ je třeba použít operátor řezu (viz dále).
\end{itemize}
\end{frame}


\begin{frame}\frametitle{Rekurze --- dobrý sluha}
{\tt
\alt<5,12>{\alert{predek(X,Y)}}{predek(X,Y)}:-\alt<6,13>{\alert{rodic(X,Y)}}{rodic(X,Y)}.\pause\\
\alt<8>{\alert{predek(X,Y)}}{predek(X,Y)}:-\alt<9>{\alert{rodic(Z,Y),predek(X,Z)}}{rodic(Z,Y),predek(X,Z)}.
}\pause
\begin{block}{}
\begin{center}
{\tt predek(adam,tubal)}
\end{center}
\end{block}\pause
{\tiny\tt
Cíl: predek(adam,tubal)\pause\\
\hskip0.5cm  predek(adam,tubal) = \alt<5>{\alert{predek(X,Y) / [X=adam,Y=tubal]}}{predek(X,Y) / [X=adam,Y=tubal]}\pause\\
\hskip1cm    Cíl: \alt<6>{\alert{rodic(X,Y) / [X=adam,Y=tubal]}}{rodic(X,Y) / [X=adam,Y=tubal]}\pause\\
\hskip1.5cm    \alert{FAIL}\pause\\
\hskip0.5cm  predek(adam,tubal) = \alt<8>{\alert{predek(X,Y) / [X=adam,Y=tubal]}}{predek(X,Y) / [X=adam,Y=tubal]}\pause\\
\hskip1cm    Cíl: \alt<9>{\alert{rodic(Z,Y), predek(X,Z) / [X=adam,Y=tubal]}}{\alt<10>{\textcolor{blue}{rodic(Z,Y)}}{\textcolor{green}{rodic(Z,Y)}}, 
                                                                              \alt<12-14>{\textcolor{blue}{predek(X,Z)}}{\alt<15->{\textcolor{green}{predek(X,Z)}}{predek(X,Z)}}
                                                                              / [X=adam,Y=tubal]}\pause\\
\hskip1.5cm    rodic(Z,Y) = rodic(kain,tubal) / [Z=kain,Y=tubal,X=adam]\pause\\
\hskip2cm         \textcolor{green}{O.K.}\pause\\
\hskip1.5cm    predek(X,Z) = \alt<12>{\alert{predek(X,Y1) / [ Z=kain,X=adam, Y1=Z, Y=tubal ]}}{predek(X,Y1) / [ Z=kain,X=adam, Y1=Z, Y=tubal ]}\pause\\
\hskip2cm         Cíl: \alt<13>{\alert{rodic(X,Y1) / [ Z=kain,X=adam, Y1=Z, Y=tubal ]}}{rodic(X,Y1) / [ Z=kain,X=adam, Y1=Z, Y=tubal ]}\pause\\
\hskip2.5cm          rodic(X,Y1) = rodic(adam,kain)\pause\\
\hskip3cm	       \textcolor{green}{O.K.}\pause\\
\hskip1.5cm	       \textcolor{green}{ÚSPĚCH}
}
\end{frame}
\begin{frame}\frametitle{Rekurze --- zlý pán}
 {\tt
\alt<5,10,15>{\alert{predek(X,Y)}}{predek(X,Y)}:-\alt<6,11,16>{\alert{rodic(X,Y)}}{rodic(X,Y)}.\pause\\
\alt<8,13,18>{\alert{predek(X,Y)}}{predek(X,Y)}:-\alt<9,14,19>{\alert{predek(X,Z),rodic(Z,Y)}}{predek(X,Z),rodic(Z,Y)}.
}\pause

\begin{block}{}
\begin{center}
{\tt predek(adam,tubal)}
\end{center}
\end{block}\pause
{\tiny\tt
Cíl: predek(adam,tubal)\pause\\
\hskip0.5cm  predek(adam,tubal) = predek(X,Y) / [X=adam,Y=tubal]\pause\\
\hskip1cm    Cíl: rodic(X,Y) / [X=adam,Y=tubal]\pause\\
\hskip1.5cm    \alert{FAIL}\pause\\
\hskip0.5cm  predek(adam,tubal) = predek(X,Y) / [X=adam,Y=tubal]\pause\\
\hskip1cm    Cíl: predek(X,Z), rodic(Z,Y) / [X=adam,Y=tubal]\pause\\
\hskip1.5cm    predek(X,Z) = predek(X,Y1) / [X=adam,Y=tubal]\pause\\
\hskip2cm        Cíl: rodic(X,Y1) / [X=adam,Y=tubal]\pause\\
\hskip2.5cm	  \alert{FAIL}\pause\\
\hskip1.5cm    predek(X,Z) = predek(X,Y1) / [X=adam,Y=tubal]\pause\\
\hskip2cm        Cíl: predek(X,Z1), rodic(Z1,Y) / [X=adam,Y=tubal]\pause\\
\hskip2.5cm        predek(X,Z1) = predek(X,Y1) / [ X=adam, Y=tubal ]\pause\\
\hskip3cm            Cíl: rodic(X,Y1) / [ X=adam,Y=tubal ]\pause\\
\hskip3.5cm	       \alert{FAIL}\pause\\
\hskip2.5cm        predek(X,Z1) = predek(X,Y1) / [ X=adam, Y=tubal ]\pause\\
\hskip3cm            Cíl: predek(X,Z2),rodic(Z2,Y1) / [ X=adam,Y=tubal ]\pause\\
\hskip3.5cm		\alert{$\ldots$}
}
\end{frame}
\begin{frame}\frametitle{Seznamy}
\begin{itemize}
 \item {\tt [1,2,3,4,5]}\pause
 \item {\tt [H|T]}\pause
\end{itemize}
 \vskip0.7cm
\alert{Příklad:} Predikát býti prvkem.\pause\\
{\tt
 contains(X,[X]).\pause\\
 contains(X,[X|\_]).\pause\\
 contains(X,[\_|T]):-contains(X,T).\pause\\
 }
 \vskip0.7cm
 \alert{Cvičení}\pause
\begin{itemize}
 \item[1] Predikát konkatenace dvou seznamů.\pause
 \item[2] Predikát otočení seznamu.\pause
 \item[3] Predikát vybrání lichých prvků.
\end{itemize}
\end{frame}

\begin{frame}\frametitle{Operátor řezu}
\begin{itemize}
 \item predikát sourozenci nebyl správně\pause
 \item chtěli bychom klauzuli zajišťující $X \neq Y $\pause
 \item lze zařídit pomocí operátoru řezu\pause\ a predikátu {\tt fail}\pause
\end{itemize}
{\tt
  test(X,Y):-test1(X,Y),!,test2(X,Y)\pause\\
}
\vskip0.7cm
\begin{center}
Význam\pause
\end{center}
\begin{itemize}
 \item nebacktrackuj přes {\tt !}\pause
 \item t.j. pokud v téhle větvi neuspěješ, nesnaž se hlavu splnit jinak\pause
\end{itemize}
\vskip0.4cm
{\tt
  different(X,X):-!,fail.\pause\\
  different(\_,\_).\pause\\
  sourozenec(X,Y) :- rodic(R,X), rodic(R,Y), different(X,Y).\pause\\
}
\end{frame}

\begin{frame}\frametitle{Robinsonova Aritmetika}

\begin{itemize}
 \item[1] numerály (za pomoci $0$, $s$)
 \item[2] $+,*$
 \item[3] $\leq$
 \item[4] prvočíselnost
 \item[5] rozklad na prvočinitele
\end{itemize}
\end{frame}

\begin{frame}\frametitle{NLP --- Natural Language Processing}
\begin{itemize}
 \item Rozdílové seznamy
 \item Sentence --> NounPhrase VerbPhrase
 \item NounPhrase --> Adjective1, Adjective2, ..., Adjectiven, Noun
 \item VerbPhrase --> Verb
 \item VerbPhrase --> Verb NounPhrase
\end{itemize}
\end{frame}


\begin{frame}\frametitle{Více $\ldots$}
\begin{center}
 \emph{Výuka}\pause
\end{center}
P. Švarný: Logical Programming\pause
\vskip1cm
\begin{center}
 \emph{Internetové zdroje}\pause
\end{center}
{\tt
 \url{http://www.learnprolognow.org/}\pause\
 \url{http://www.amzi.com/AdventureInProlog/}\pause\
 \url{http://ksvi.mff.cuni.cz/\~kryl/prolog.pdf}\pause\
 }
 \begin{center}
   \emph{Knižní zdroje}\pause
 \end{center}
 Jirků, P. a kol.: Programování v jazyku Prolog, Praha 1991\\
 Ivan Bratko: Prolog Programming for Artificial Intelligence, 1986\\
\end{frame}




\end{document}